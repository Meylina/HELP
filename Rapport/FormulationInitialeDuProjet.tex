\chapter{Formulation initiale du projet}



\section{Contexte}

Dans le cadre de l'U.V. 3.4, notre équipe a choisi de développer le projet HELP (Hope to Emulate the Life of Paralyzed people) proposé par Mr Mansour. Ce projet a pour objectif la réalisation d'un système permettant de remplacer la souris d’ordinateur grâce aux mouvements des yeux. Cette étude semble très intéressante car elle demande une analyse et une compréhension de systèmes complexes d'eye tracking (oculométrie) déjà existant afin de mettre au point une version simplifiée et moins onéreuse de ces systèmes. De plus, elle réunit différents aspects du travail d'ingénieur en informatique tels que le traitement de l'image, l'algorithmique, le travail en équipe,... Enfin, ce projet peut éventuellement mener à deux finalités différentes : d'abord, l'aide aux personnes tétraplégiques, qui, grâce à ce système, pourraient être moins dépendantes et retrouver un peu de liberté. Ensuite, ce projet pourrait aussi être utilisé afin d'aider les scientifiques à mettre au point un robot travaillant en zones hostiles qui puisse être contrôlé facilement grâce à la détection des mouvements de la tête et des yeux de l'opérateur.

\section{Expression initiale du besoin}

Le but premier de ce projet était le développement d'un système permettant à une personne tétraplégique d'utiliser un ordinateur et surfer sur internet. Cependant, face à l'ampleur du projet, et suite à un entretien avec nos encadrant, nous avons décidé de commencer par développer un système permettant à une personne ordinaire de contrôler un ordinateur. Ainsi, le système développé doit permettre à un utilisateur d'exécuter différentes applications sans avoir besoin de toucher une souris ou un clavier. L'utilisateur doit pouvoir effectuer les opérations usuelles en bougeant et clignant des yeux.

\section*{Conclusion}

La première partie de ce projet nous a permis de comprendre les étapes qui mènent à l’aboutissement d’un projet. En effet, afin de se faire une idée de ce qui est réalisable, nous avons tout d’abord procédé à un état de l’art. Cette étude préalable nous a permis de comprendre que de nombreuses technologies existent déjà afin d’effectuer de l’eye tracking. A la fin de cette première partie de projet, notre choix s’est orienté vers un système composé de deux caméras. Cependant lors de la réalisation de se système, nous avons rencontré certaines difficultés qui nous ont emmené à redéfinir et simplifier notre système. Nous nous sommes ainsi rendu compte de l’importance des phases d’expérimentation et de conception, notamment pour un projet d’une telle envergure. 
\bigbreak
De plus, si la première partie du projet (partie IS) est importante et que la partie sur l’état de l’art permet d’étudier un panel de solution envisageable à notre système, celui-ci n’a été réellement définis que lors de la phase de conception. En effet, cette phase nous a permis tester certaines idées, de faire des choix plus adaptés à nos compétences et d’améliorer, raffiner la partie IS du projet.
\bigbreak
Enfin, si nous avons revus nos exigences à la baisse durant ce projet, nous avons tout de même un démonstrateur fonctionnel et relativement robuste. De plus, la réalisation de celui-ci nous a permis de nous plonger dans le domaine du gaze-tracking encore peu exploré à ce jour. Cela nous a également permis de développer nos compétences en programmation, notamment en openCV et en C++.
